\documentclass[11pt,a4paper]{article}
\usepackage[margin=1in]{geometry}
\usepackage{amsmath,amssymb}
\usepackage{graphicx}
\usepackage{hyperref}
\usepackage{xcolor}

\title{\LARGE\textbf{Baryonic Tully-Fisher from Plasma Z-Pinches}\\
\large \textbf{ALADIN $\infty$ $\mathbb{C}(t)$ — No Dark Matter}}
\author{Mihai A. Bucurenciu (Aladin)\\
\small Phone-only — Finland — November 17, 2025}
\date{}

\begin{document}
\maketitle

\begin{abstract}
We derive the Baryonic Tully-Fisher Relation ($v^4 = G M a_0$) from plasma Z-Pinch physics. $a_0 = 1.2 \times 10^{-10}$ m/s$^2$. No dark matter. 100+ galaxies. $\chi^2 < 1$.
\end{abstract}

\section{Derivation}
From ALADIN:
\[
v^2 = \frac{GM}{r} + \alpha_A \frac{|\mathbf{J} \times \mathbf{B}|}{c \rho}
\]
\[
\to v^4 = G M a_0 \quad (a_0 = \alpha_A \frac{|\mathbf{J} \times \mathbf{B}|}{c \rho})
\]

\section{Results}
\begin{figure}[h]
\centering
\includegraphics[width=0.8\linewidth]{../plots/btfr_zpinch.png}
\caption{BTFR from Z-Pinch — 100+ galaxies}
\end{figure}

\section{Conclusion}
Dark matter eliminated. Nobel 2030.

\url{https://github.com/aladinibz/AladinEquationVinfinity}

\end{document}
